\documentclass{article}

\usepackage[margin=3cm]{geometry}
\usepackage[fleqn]{amsmath}
\usepackage{amsthm}
\usepackage{mathtools}

% new theorem-like environment
\newtheorem{proposition}{Proposition}

\begin{document}




% question ---------------------------------------------------------------------

\noindent \textbf{Exercise 1.13:} Prove that
\begin{equation*}
  \mathit{Fib}(n) \text{ is the closest integer to }
  \frac{ \phi^n }{ \sqrt{5} }, \text{ where } \phi = \frac{1 + \sqrt{5}}{2}.
\end{equation*}
Hint: let $\Psi = \frac{1 - \sqrt{5}}{2}$.  Use induction and the definition of
the Fibonacci numbers (see section 1.2.2) to prove that
\begin{equation*}
  \mathit{Fib}(n) = \frac{\phi^n - \Psi^n}{\sqrt{5}}.
\end{equation*}

\hrule
\vspace{10mm}




% proposition 1 ----------------------------------------------------------------

We begin with a pair of identities which will be of use later.  Recall that
Proposition \ref{phi-identity} was provided in Section 1.2.2.

\begin{proposition}
  \label{phi-identity}
  $\phi^2 = \phi + 1$.
\end{proposition}

\begin{proof}
  We have
  \begin{equation*}
    \phi^2
    = \left( \frac{ 1 + \sqrt{5} }{ 2 } \right)^2
    = \frac{ 1 + 2\sqrt{5} + 5 }{ 4 }
    = \frac{ 3 + \sqrt{5} }{ 2 },
  \end{equation*}
  and
  \begin{equation*}
    1 + \phi
    = 1 + \frac{ 1 + \sqrt{5} }{ 2 }
    = \frac{ 3 + \sqrt{5} }{ 2 }.
  \end{equation*}
\end{proof}
\vspace{5mm}




% proposition 2 ----------------------------------------------------------------

\begin{proposition}
  $\Psi^2 = \Psi + 1$.
\end{proposition}

\begin{proof}
  We have
  \begin{equation*}
    \Psi^2
    = \left( \frac{ 1 - \sqrt{5} }{ 2 } \right)^2
    = \frac{ 1 - 2\sqrt{5} + 5 }{ 4 }
    = \frac{ 3 - \sqrt{5} }{ 2 },
  \end{equation*}
  and
  \begin{equation*}
    1 + \Psi
    = 1 + \frac{ 1 - \sqrt{5} }{ 2 }
    = \frac{ 3 - \sqrt{5} }{ 2 }.
  \end{equation*}
\end{proof}
\vspace{5mm}




% proposition 3 ----------------------------------------------------------------

\begin{proposition}
  $\displaystyle \mathit{Fib}(n) = \frac{\phi^n - \Psi^n}{\sqrt{5}}.$
\end{proposition}

\begin{proof}
  We are going to use induction and make use of the definition of the Fibonacci
  sequence (in this case
  $\mathit{Fib}(n + 2) = \mathit{Fib}(n + 1) + \mathit{Fib}(n)$) as the hint
  suggests.  Since we need to look back two terms for induction, we need to
  establish two base cases.

  \paragraph{First base case: $n = 0$.}
  \begin{equation*}
    \frac{\phi^0 - \Psi^0}{\sqrt{5}} = \frac{1 - 1}{\sqrt{5}} = 0 = \mathit{Fib}(0).
  \end{equation*}

  \paragraph{Second base case: $n = 1$.}
  \begin{equation*}
    \frac{\phi^1 - \Psi^1}{\sqrt{5}}
    = \frac{ 1 }{ \sqrt{5} }
    \left[
      \frac{ 1 + \sqrt{5} }{ 2 } - \frac{ 1 - \sqrt{5} }{ 2 }
    \right]
    = \frac{ 1 }{ \sqrt{5} }\, \frac{ 2 \sqrt{5} }{ 2 }
    = 1
    = \mathit{Fib}(1).
  \end{equation*}

  \paragraph{Induction step.}  Note that although for convenience we will work
  in the opposite direction as we did for the base cases, since we are dealing
  with equalities they are of course equivalent.
  \begin{align*} \MoveEqLeft
    \mathit{Fib}(n + 2) = \mathit{Fib}(n + 1) + \mathit{Fib}(n) \\[1ex]
    &= \frac{\phi^{n + 1} - \Psi^{n + 1}}{\sqrt{5}}
      + \frac{\phi^n - \Psi^n}{\sqrt{5}} \\[1ex]
    &= \frac{ 1 }{ \sqrt{5} }
      \Big[
      \phi^{n + 1} + \phi^n - (\Psi^{n + 1} + \Psi^n)
      \Big] \\[1ex]
    &= \frac{ 1 }{ \sqrt{5} }
      \Big[
      \phi^n(\phi + 1) - \Psi^n(\Psi + 1)
      \Big] \\[1ex]
    &= \frac{ 1 }{ \sqrt{5} }
      \Big[
      \phi^n\phi^2 - \Psi^n\Psi^2
      \Big] && \text{(by Proposition 1 and 2)} \\[1ex]
    &= \frac{\phi^{n + 2} - \Psi^{n + 2}}{\sqrt{5}}
  \end{align*}

\end{proof}
\vspace{5mm}




% proposition 4 ----------------------------------------------------------------

Now we will go ahead and prove the main clam.  The way that this is achieved is
by showing that the difference between $\mathit{Fib}(n)$ and the approximation
$\frac{ \phi^n }{ \sqrt{5} }$ is always less than $\frac{1}{2}$, from which the
claim immediately follows.

\begin{proposition}
  $\displaystyle \left| \mathit{Fib}(n) - \frac{ \phi^n }{ \sqrt{5} } \right| < \frac{1}{2}$.
\end{proposition}

\begin{proof}
  We first note that
  \begin{equation*}
    \Psi = \frac{ 1 - \sqrt{5} }{ 2 } < \frac{ 1 - \sqrt{4} }{ 2 } = - \frac{1}{2},
  \end{equation*}
  and that
  \begin{equation*}
    \Psi = \frac{ 1 - \sqrt{5} }{ 2 } > \frac{ 1 - \sqrt{9} }{ 2 } = -1,
  \end{equation*}
  so it follows that
  \begin{equation*}
    -1 < \Psi^n < 1.
  \end{equation*}
  Next we have
  \begin{equation*}
    \mathit{Fib}(n) - \frac{ \phi^n }{ \sqrt{5} }
    = - \frac{ \Psi^n }{ \sqrt{5} }
    < \frac{ 1 }{ \sqrt{5} }
    < \frac{ 1 }{ \sqrt{4} }
    = \frac{ 1 }{ 2 },
  \end{equation*}
  and similarly
  \begin{equation*}
    \mathit{Fib}(n) - \frac{ \phi^n }{ \sqrt{5} }
    = - \frac{ \Psi^n }{ \sqrt{5} }
    > - \frac{ 1 }{ \sqrt{5} }
    > - \frac{ 1 }{ \sqrt{4} }
    = - \frac{ 1 }{ 2 }.
  \end{equation*}
  Putting these last two inequalities together gives us
  \begin{equation*}
    -\frac{1}{2} < \mathit{Fib}(n) - \frac{ \phi^n }{ \sqrt{5} } < \frac{1}{2},
  \end{equation*}
  which is equivalent to the desired form.
\end{proof}




\end{document}

%%% Local Variables:
%%% mode: latex
%%% TeX-master: t
%%% End:
