\documentclass{article}

\usepackage[margin=3cm]{geometry}
\usepackage[fleqn]{amsmath}
\usepackage{amsthm}
\usepackage{amsfonts}
% \usepackage{mathtools}
\usepackage{listings}
\usepackage[x11names]{xcolor}

% define a global font for listings environments
\lstset{basicstyle=\fontsize{9.5}{10.5}\fontfamily{pcr}\selectfont}

% define a listings environment for Scheme
\lstdefinestyle{scheme}{%
  frame=lines,
  language=lisp,
  basicstyle=\fontsize{9.5}{10.5}\fontfamily{pcr}\selectfont,
  keywordstyle=\color{Blue1},
  commentstyle=\color{DarkOrchid4},
  morekeywords={define},
  deletekeywords={count}
}

% new theorem-like environment
\newtheorem{proposition}{Proposition}




\begin{document}


% question ---------------------------------------------------------------------

\noindent \textbf{Exercise 1.19:} There is a clever algorithm for computing the
Fibonacci numbers in a logarithmic number of steps.  Recall the transformation
of the state variables $a$ and $b$ in the \lstinline{fib-iter} process of
section 1.2.2: $a \leftarrow a + b$ and $b \leftarrow a$.  Call this
transformation $T$, and observe that applying $T$ over and over again $n$ times,
starting with 1 and 0, produces the pair $\mathit{Fib}(n + 1)$ and
$\mathit{Fib}(n)$.

In other words, the Fibonacci numbers are produced by applying $T^n$, the $n$-th
power of the transformation $T$, starting with the pair $(1,0)$.  Now consider
$T$ to be the special case of $p = 0$ and $q = 1$ in a family of transformations
$T_{pq}$, where $T_{pq}$ transforms the pair $(a,b)$ according to
$a \leftarrow bq + aq + ap$ and $b \leftarrow bp + aq$.  Show that if we apply
such a transformation $T_{pq}$ twice, the effect is the same as using a single
transformation $T_{p^{\prime}q^{\prime}}$ of the same form, and compute
$p^{\prime}$ and $q^{\prime}$ in terms of $p$ and $q$.  This gives us an
explicit way to square these transformations, and thus we can compute $T^n$
using successive squaring, as in the \lstinline{fast-expt} procedure.  Put this
all together to complete the following procedure, which runs in a logarithmic
number of steps:

\vspace{5mm}
\begin{lstlisting}[style=scheme]
(define (fib n)
  (fib-iter 1 0 0 1 n))

(define (fib-iter a b p q count)
  (cond ((= count 0) b)
        ((even? count)
         (fib-iter a
                   b
                   <??>      ; compute p'
                   <??>      ; compute q'
                   (/ count 2)))
        (else (fib-iter (+ (* b q) (* a q) (* a p))
                        (+ (* b p) (* a q))
                        p
                        q
                        (- count 1)))))
\end{lstlisting}

\vspace{10mm}



% begin solution ---------------------------------------------------------------

\noindent To begin with, we want to find a function
$T: \mathbb{R}^2 \mapsto \mathbb{R}^2$ such that $T(a, b) = (a + b, a)$.  It is
easy to intuit that $T$ is the linear operator defined by
\begin{equation*}
  T =
  \begin{bmatrix}
    1 & 1 \\
    1 & 0
  \end{bmatrix},
\end{equation*}
and we can write $T(a, b)$ in the form
\begin{equation*}
  T
  \begin{bmatrix}
    a \\
    b
  \end{bmatrix}
  =
  \begin{bmatrix}
    a + b \\
    b
  \end{bmatrix}.
\end{equation*}
Next, Proposition \ref{fast-iter-proof} formally states the procedure performed
by the \lstinline{fib-iter} process of section 1.2.2.




% proposition 1 ----------------------------------------------------------------

\vspace{10mm}
\begin{proposition}
  \label{fast-iter-proof}
  Let $T: \mathbb{R}^2 \mapsto \mathbb{R}^2$ be the linear operator defined by
  \begin{equation*}
    T =
    \begin{bmatrix}
      1 & 1 \\
      1 & 0
    \end{bmatrix},
  \end{equation*}
  and let us write $F_k$ as a shorthand for $\mathit{Fib}(k), k = 0, 1, \dots$.
  Then
  \begin{equation*}
    T^n
    \begin{bmatrix}
      F_1 \\
      F_0
    \end{bmatrix}
    =
    \begin{bmatrix}
      F_{n + 1} \\
      F_n
    \end{bmatrix}
  \end{equation*}
\end{proposition}

\vspace{5mm}
\begin{proof}
  We proceed with proof by induction.
  \paragraph{Base case:} $n = 1$.
  \begin{equation*}
    T^1
    \begin{bmatrix}
      F_1 \\
      F_0
    \end{bmatrix}
    =
    \begin{bmatrix}
      1 & 1 \\
      1 & 0
    \end{bmatrix}
    \begin{bmatrix}
      F_! \\
      F_0
    \end{bmatrix}
    =
    \begin{bmatrix}
      F_1 + F_0 \\
      F_1
    \end{bmatrix}
    =
    \begin{bmatrix}
      F_2 \\
      F_1
    \end{bmatrix}
  \end{equation*}

  \paragraph{Induction step:} Assume that the claim holds for the $(n - 1)$-th
  case.  Then
  \begin{equation*}
    T^n
    \begin{bmatrix}
      F_1 \\
      F_0
    \end{bmatrix}
    =
    T T^{n - 1}
    \begin{bmatrix}
      F_1 \\
      F_0
    \end{bmatrix}
    =
    T
    \begin{bmatrix}
      F_{n} \\
      F_{n - 1}
    \end{bmatrix}
    =
    \begin{bmatrix}
      1 & 1 \\
      1 & 0
    \end{bmatrix}
    \begin{bmatrix}
      F_{n} \\
      F_{n - 1}
    \end{bmatrix}
    =
    \begin{bmatrix}
      F_{n} + F_{n - 1} \\
      F_n
    \end{bmatrix}
    =
    \begin{bmatrix}
      F_{n + 1} \\
      F_n
    \end{bmatrix}.
  \end{equation*}

\end{proof}
\vspace{5mm}




% ------------------------------------------------------------------------------

\noindent Next, we are directed to find a function
$T_{pq}: \mathbb{R}^2 \mapsto \mathbb{R}^2$ such that
$T_{pq}(a, b) = (bq + aq + ap,~ bp + aq)$.  The reason that we want to find the
form of this particular family of operators is described as follows.

Recall from Proposition \ref{fast-iter-proof} that we can calculate the
$(n + 1)$-th value of the Fibonacci sequence by calculating $T^n (F_1, F_0)$.
In the \lstinline{fib-iter} process of section 1.2.2 we performed this
calculation by applying the $T$ operator $n$ times.  However the goal of the
logarithmic Fibonacci algorithm is to successively square terms of $T$ as in
e.g.  $((T^2)^2)^2$, when possible.  In general, we can track the value of any
$2 \times 2$ matrix such as those obtained from calculating $T^k$ for some
$k \in \mathbb{N}$ using four variables (e.g. $t_{11}, t_{12}, t_{21}, t_{22}$),
but as it turns out, we can track the values of $T^k$ using only 2 parameters,
$p$ and $q$.  This statement is formalized in Proposition
\ref{thm:parameterize-power-t}.


\vspace{10mm}
\begin{proposition}
  \label{thm:parameterize-power-t}
  Let
  \begin{equation*}
    T =
    \begin{bmatrix}
      1 & 1 \\
      1 & 0
    \end{bmatrix}.
  \end{equation*}
  Then there exist $p, q \in \{0\} \cup \mathbb{N}$ such that
  \begin{equation*}
    T^k =
    \begin{bmatrix}
      p + q & q \\
      q     & p
    \end{bmatrix},
  \end{equation*}
  for $k \in \mathbb{N}$.
\end{proposition}

\vspace{5mm}
\begin{proof}
  We proceed with proof by induction.
  \paragraph{Base case:} $n = 1$.
  \begin{equation*}
    T^1 =
    \begin{bmatrix}
      1 & 1 \\
      1 & 0
    \end{bmatrix},
  \end{equation*}
  so by letting $p = 0$ and $q = 1$ we have the desired form.

  \paragraph{Induction step:} Assume that the claim holds for the $k$-th case.
  Then
  \begin{equation*}
    T^{k + 1} = TT^k = T
    \begin{bmatrix}
      p + q & q \\
      q     & p
    \end{bmatrix}
    =
    \begin{bmatrix}
      p + 2q & p + q \\
      p + q  & q
    \end{bmatrix},
  \end{equation*}
  so that by letting $p^{\prime} = q$ and $q^{\prime} = p + q$ we see that
  \begin{equation*}
    T^{k + 1} =
    \begin{bmatrix}
      p^{\prime} + q^{\prime} & q^{\prime} \\
      q^{\prime}             & p^{\prime}
    \end{bmatrix},
  \end{equation*}
  and we have the desired form.
\end{proof}
\vspace{5mm}

Now that we have some intuition as to why we are considering a family of
transformations $\{T_{pq}\}$, we can return to the problem of finding an
operator $T_{pq}: \mathbb{R}^2 \mapsto \mathbb{R}^2$ such that
$T_{pq}(a, b) = (bq + aq + ap,~ bp + aq)$.  If we posit that for fixed $p, q$
then $T_{pq}$ is a linear operator, then we can find the operator by solving a
system of equations, since it turns out to indeed be linear.  Furthermore, the
family of operators $\{T_{pq}\}$ are those operators described in Proposition
\ref{thm:parameterize-power-t}, as stated below.

% \vspace{40mm}
% We were told that $T$ is a special
% case of $T_{pq}$, so if we make a guess that for fixed values of $p$ and $q$
% then $T_{pq}$ is a linear operator, then we can attempt to solve the system of
% equations given by
\begin{proposition}
  \label{thm:t-pq}
  Let $T_{pq}$ be defined as
  \begin{equation*}
    T_{pq} =
    \begin{bmatrix}
      p + q & q \\
      q     & p
    \end{bmatrix},
  \end{equation*}
  then
  \begin{equation*}
    T_{pq}(a, b) = (bq + aq + ap,~ bp + aq),
  \end{equation*}
  which is the statement in the problem prompt.
\end{proposition}
\vspace{5mm}

\begin{proof}
  We have
  \begin{equation*}
    T_{pq}
    \begin{bmatrix}
      a \\
      b
    \end{bmatrix}
    =
    \begin{bmatrix}
      p + q & q \\
      q     & p
    \end{bmatrix}
    \begin{bmatrix}
      a \\
      b
    \end{bmatrix}
    =
    \begin{bmatrix}
      pa + qa + qb \\
      qa + pb
    \end{bmatrix},
  \end{equation*}
  which is equivalent to the form stated in the claim.
\end{proof}
\vspace{5mm}


Next, we are directed to find the parameterization of the operator that results
from squaring $T_{pq}$.  This is the topic of Proposition \ref{thm:t-square}.
\begin{proposition}
  \label{thm:t-square}
  The operator that results from squaring $T_{pq}$ is given by
  \begin{equation*}
    T_{pq}^2 = T_{p^{\prime} q^{\prime}},
  \end{equation*}
  where $p^{\prime} = p^2 + q^2$ and $q^{\prime} = 2pq + q^2$.
\end{proposition}
\vspace{5mm}

\begin{proof}
  \begin{equation*}
    T_{pq}^2 =
    \begin{bmatrix}
      p + q & q \\
      q     & p
    \end{bmatrix}
    \begin{bmatrix}
      p + q & q \\
      q     & p
    \end{bmatrix}
    =
    \begin{bmatrix}
      (p + q)^2 + q^2 & 2pq + q^2 \\
      2pq + q^2        & p^2 + q^2
    \end{bmatrix}
    =
    \begin{bmatrix}
      p^{\prime} + q^{\prime} & q^{\prime} \\
      q^{\prime}             & p^{\prime}
    \end{bmatrix},
  \end{equation*}
  where $p^{\prime} = p^2 + q^2$ and $q^{\prime} = 2pq + q^2$.  Since this is
  the form of the family of operators $\{T_{pq}\}$, the claim follows.
\end{proof}
\vspace{5mm}

Now that we have all of the pieces in place, we want to describe an approach for
calculating the $(n + 1)$-th term in the Fibonacci sequence.  The main idea is
to consider the expression $T_{01}^n(F_1, F_0) = (F_{n + 1}, F_n)$, and at each
step of the algorithm to consider an equivalent form of the expression with a
reduced value of $n$.  We consider the following two identities.  When $n$ is
even, then we consider the identity
\begin{equation*}
  T_{pq}^n
  \begin{bmatrix}
    a \\
    b
  \end{bmatrix}
  = \left(T_{pq}^2\right)^{n / 2}
  \begin{bmatrix}
    a \\
    b
  \end{bmatrix}
  =
  T_{p^{\prime} q^{\prime}}^{n / 2}
  \begin{bmatrix}
    a \\
    b
  \end{bmatrix},
\end{equation*}
and when $n$ is odd, then we consider the identity
\begin{equation*}
  T_{pq}^n
  \begin{bmatrix}
    a \\
    b
  \end{bmatrix}
  = T_{pq}^{n - 1} T_{pq}
  \begin{bmatrix}
    a \\
    b
  \end{bmatrix}
  = T_{pq}^{n - 1}
  \begin{bmatrix}
    a^{\prime} \\
    b^{\prime}
  \end{bmatrix},
\end{equation*}
Thus we can construct an algorithm that at each step converts the data according
to one of these two identities, and terminates when $n = 0$.  Since $n$ is
always decreasing the algorithm is guaranteed to terminate, and since the value
of the expression is unchanged at each step of the algorithm, we are guaranteed
to obtain the correct result.

The implementation of this procedure is provided below (where
\lstinline{count} tracks the value of $n$).

\vspace{5mm}
\begin{lstlisting}[style=scheme]
(define (fib n)
  (fib-iter 1 0 0 1 n))

(define (fib-iter a b p q count)
  (cond ((= count 0) b)
        ((even? count)
         (fib-iter a
                   b
                   (+ (square p) (square q))
                   (+ (* 2 p q) (square q))
                   (/ count 2)))
        (else (fib-iter (+ (* b q) (* a q) (* a p))
                        (+ (* b p) (* a q))
                        p
                        q
                        (- count 1)))))
\end{lstlisting}




\end{document}

%%% Local Variables:
%%% mode: latex
%%% TeX-master: t
%%% End:
