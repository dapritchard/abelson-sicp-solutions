\documentclass{article}

\usepackage[margin=3cm]{geometry}
\usepackage[fleqn]{amsmath}
\usepackage{amsthm}
\usepackage{amsfonts}
\usepackage{mathtools}
\usepackage{listings}
\usepackage[x11names]{xcolor}

% define a global font for listings environments
\lstset{basicstyle=\fontsize{9.5}{10.5}\fontfamily{pcr}\selectfont}

% define a listings environment for Scheme
\lstdefinestyle{scheme}{%
  frame=lines,
  language=lisp,
  basicstyle=\fontsize{9.5}{10.5}\fontfamily{pcr}\selectfont,
  keywordstyle=\color{Blue1},
  commentstyle=\color{DarkOrchid4},
  morekeywords={define, if},
  deletekeywords={count, gcd}
}

% new theorem-like environment
\newtheorem{proposition}{Proposition}
\newtheorem{theorem}{Theorem}

\DeclareMathOperator{\GCD}{GCD}
\DeclareMathOperator{\fib}{Fib}




\begin{document}


% question ---------------------------------------------------------------------

\noindent \textbf{Exercise 1.20:} The process that a procedure generates is of
course dependent on the rules used by the interpreter.  As an example, consider
the iterative \lstinline{gcd} procedure given below.  Suppose we were to
interpret this procedure using normal-order evaluation, as discussed in section
1.1.5.  (The normal-order-evaluation rule for \lstinline{if} is described in
Exercise 1.5.)  Using the substitution method (for normal order), illustrate the
process generated in evaluating \lstinline{(gcd 206 40)} and indicate the
\lstinline{remainder} operations that are actually performed.  How many
\lstinline{remainder} operations are actually performed in the normal-order
evaluation of \lstinline{(gcd 206 40)}?  In the applicative-order evaluation?
\vspace{3mm}
\begin{lstlisting}[style=scheme]
(define (gcd a b)
  (if (= b 0)
      a
      (gcd b (remainder a b))))
\end{lstlisting}
\vspace{10mm}




% preamble ---------------------------------------------------------------------

Before we answer this question, we do some work to establish the claim that
$\GCD(a,b) = \GCD(b,r)$, and that the number of steps required by the algorithm
has order $\mathcal{O}(\log b)$.

We note however that we do not establish the stronger claim that the number of
steps required by the algorithm has order $\Theta(\log b)$, as it is stated in
the book, since it seems that the lower bound for the number of steps is 1 (for
example, when $a = b$).  We also note that this isn't a result for the overall
complexity of the algorithm, since it doesn't account for the cost of actually
calculating $a \bmod b$ for each step, which grows as a function of $a$ and $b$.

\vspace{5mm}




% GCD equivalency --------------------------------------------------------------

\begin{proposition}
  Suppose $a, b \in \mathbb{N}$ with $a \geq b$.  If $r$ is the remainder when
  $a$ is divided by $b$, then $\GCD(a,b) = \GCD(b,r)$.
\end{proposition}

\begin{proof}
  Note that there exists some $q \in \mathbb{N}$ such that $a = qb + r$.
  Suppose that $\GCD(a, b) = k$.  Since $b \bmod k = 0$, it follows that
  $qb \bmod k = 0$.  Then, because of this fact and since $a \bmod k = 0$, it
  follows that $r \bmod k = 0$.  This shows that $k$ is a common divisor of $b$
  and $r$, and consequently we have that $k \leq \GCD(b, r)$.

  Now we want to show that there is no larger common divisor than $k$.  Suppose
  that the converse is true, so that $\GCD(b, r) = k^{\prime} > k$.  Then
  $b \bmod k^{\prime} = 0$, so it follows that $qb \bmod k^{\prime} = 0$.  Then
  because of this fact and since $r \bmod k^{\prime} = 0$ we obtain that
  $(qb + r) \bmod k^{\prime} = 0$ which is equivalent to
  $a \bmod k^{\prime} = 0$.  This would mean that $k^{\prime}$ is a common
  divisor of $a$ and $b$ from which it would follow that
  $k < k^{\prime} \leq \GCD(a, b)$.  However, this is a contradiction since
  $k = \GCD(a, b)$, and so we obtain that $\GCD(b, r) \leq k$

  Putting these two pieces together yields $k \leq \GCD(b, r) \leq k$.
\end{proof}
\vspace{5mm}




% b_{k + 1} >= b_k + b_{k - 1} -------------------------------------------------

\begin{proposition}
  \label{thm:bound-for-euclid-bs}
  If the pairs $(a_{k + 1}, b_{k + 1}),~ (a_k, b_k)$, and
  $(a_{k - 1}, b_{k - 1})$ are successive pairs obtained using Euclid's
  algorithm, then $b_{k + 1} \geq b_k + b_{k - 1}$.
\end{proposition}

\begin{proof}
  Let $f: \mathbb{R}^2 \mapsto \mathbb{R}^2$ be the function that maps
  successive pairs in Euclid's algorithm, so that $f(a, b) = (b,\, a \bmod b)$.
  Then we have that
  \begin{equation*}
    f(a_{k + 1}, b_{k + 1}) = (a_k, b_k) \hspace{5mm} \Longrightarrow \hspace{5mm} b_{k + 1} = a_k,
  \end{equation*}
  and
  \begin{equation*}
    f(a_{k}, b_{k}) = (a_{k - 1}, b_{k - 1}) \hspace{5mm} \Longrightarrow \hspace{5mm} a_{k} = qb_{k} + b_{k - 1}
  \end{equation*}
  for some $q \in \mathbb{N}$.  Substituting $b_{k + 1}$ into the second result yields
  \begin{equation*}
    b_{k + 1} = qb_{k} + b_{k - 1} \geq b_{k} + b_{k - 1},
  \end{equation*}
  since $q \geq 1$.
\end{proof}

\vspace{5mm}




% Lame's Theorem ---------------------------------------------------------------

\begin{theorem}[Lam\'e's Theorem]
  If Euclid's Algorithm requires $k$ steps to compute the greatest common
  denominator of some pair, then the smaller number in the pair must be greater
  than or equal to the $k$-th Fibonacci number.
\end{theorem}

\begin{proof}
  For the following proof, we assume that we start with a pair of integers
  $(a, b)$ such that $a \geq b \geq 1$, and we say that the algorithm terminates
  when some pair $(a^{\prime}, 0)$ is obtained.  We proceed with
  proof-by-induction.

  \paragraph{Base case.}  This is trivially true, since $\fib(1) = 1$, and by
  assumption $b \geq 1$.

  \paragraph{Induction step.}  Assume that the claim holds for all integers less
  than or equal to some value $k$ (this is the number of steps that the
  algorithm takes to terminate).  Suppose that $(a_{k + 1}, b_{k + 1})$ is an
  input to Euclid's algorithm that takes $k + 1$ steps to terminate, and let
  $(a_k, b_k)$, and $(a_{k - 1}, b_{k - 1})$ denote the first two pairs obtained
  using the algorithm, respectively.  Note that we can infer that Euclid's
  algorithm takes $k$ steps to complete when provided $(a_k, b_k)$ as input, and
  $k - 1$ steps to complete when provided $(a_{k - 1}, b_{k - 1})$ as input.

  Now, by Proposition \ref{thm:bound-for-euclid-bs}, we have
  $b_{k + 1} \geq b_k + b_{k - 1}$.  Furthermore, by the induction hypothesis,
  we have that $b_k \geq \fib(b_k)$, and $b_{k - 1} \geq \fib(b_{k - 1})$.
  Thus, it follows that
  \begin{equation*}
    b_{k + 1} \geq b_k + b_{k - 1} \geq \fib(k) + \fib(k - 1) = \fib(k +1),
  \end{equation*}
  which completes the proof.
\end{proof}
\vspace{5mm}




% bound for the number of steps ------------------------------------------------

\begin{proposition}
  Let $n$ be be the smaller of the two inputs to Euclid's algorithm.  Then
  number of steps that the algorithm requires is of the order
  $\mathcal{O}(\log n)$.
\end{proposition}

\begin{proof}
  Recall from Exercise 1.13 that
  \begin{equation*}
    \fib(k) = \frac{\phi^k - \Psi^k}{\sqrt{5}},
  \end{equation*}
  where $\phi = \frac{1}{2}(1 + \sqrt{5})$ and
  $\Psi = \frac{1}{2}(1 - \sqrt{5})$.  Also recall that it was established in
  that exercise that $-1 < \Psi < -\frac{1}{2}$ so that $-1 < \Psi^n < 1$.  Now,
  if the process takes $k$ steps, then by Lam\'e's Theorem we have
  \begin{equation*}
    \frac{\phi^k - \Psi^k}{\sqrt{5}} = \fib(k) \leq n,
  \end{equation*}
  so that rearranging terms yields
  \begin{align*}
    \phi^k &\leq \sqrt{5} n - \Psi^k \\
           &\leq \sqrt{5} n + 1 \\
           &\leq 2\sqrt{5} n,
  \end{align*}
  from which it follows that
  \begin{equation*}
    k \leq \frac{ \log n }{ \log\phi } + \frac{ 2\sqrt{5} }{ \log\phi }.
  \end{equation*}
  Since we can choose a value for $M$ (say by letting $M = 2$, for example) such
  that for large values of $n$ then $M \log n$ is larger than the right-hand
  side of the previous inequality, we conclude that the number of steps $k$ of
  Euclid's algorithm has complexity $\mathcal{O}(\log n)$.
\end{proof}




% applicative-order evaluation -------------------------------------------------


\paragraph{Applicative-order evaluation.}  Next we perform applicative-order
evaluation for the a call to \lstinline{(gcd 206 40)}.  We can see that there
are 4 evaluations of the \lstinline{remainder} procedure in total.

\vspace{3mm}
\begin{lstlisting}[style=scheme]
(gcd 206 40)
(if (= 40 0)
    206
    (gcd 40 (remainder 206 40)))
(gcd 40 (remainder 206 40))

(gcd 40 6)
(if (= 6 0)
    40
    (gcd 6 (remainder 40 6)))
(gcd 6 (remainder 40 6))

(gcd 6 4)
(if (= 4 0)
    6
    (gcd 6 (remainder 6 4)))
(gcd 6 (remainder 6 4))

(gcd 4 2)
(if (= 2 0)
    4
    (gcd 2 (remainder 4 2)))
(gcd 2 (remainder 4 2))

(gcd 2 0)
(if (= 0 0)
    2
    (gcd 0 (remainder 2 0)))
2
\end{lstlisting}
\vspace{10mm}




% normal-order evaluation ------------------------------------------------------

\paragraph{Normal-order evaluation.}  Next we perform normal-order evaluation
for the a call to \lstinline{(gcd 206 40)}.  We can see that there are 18 total
evaluations of the \lstinline{remainder} procedure: 14 that result when
evaluating the conditional, and 4 in the final reduction phase.

\vspace{3mm}
\begin{lstlisting}[style=scheme]
;; a = 206
;; b = 40
(gcd 206 40)
(if (= 40 0)
    206
    (gcd 40 (remainder 206 40)))

;; a = 40
;; b = (remainder 206 40)
(gcd 40 (remainder 206 40))
(if (= (remainder 206 40) 0)
    40
    (gcd (remainder 206 40) (remainder 40 (remainder 206 40)))))
;; have to evaluate the conditional so we need to calculate the values of the
;; operands.  Perfoms one `remainder` operation.
(if (= 6 0)
    40
    (gcd (remainder 206 40) (remainder 40 (remainder 206 40)))))

;; a = (remainder 206 40)
;; b = (remainder 40 (remainder 206 40))
(gcd (remainder 206 40) (remainder 40 (remainder 206 40)))
(if (= (remainder 40 (remainder 206 40)) 0)
    (remainder 206 40)
    (gcd (remainder 40 (remainder 206 40)
         (remainder (remainder 206 40)
         (remainder 40 (remainder 206 40))))))
;; have to evaluate the conditional so we need to calculate the values of the
;; operands.  Performs two `remainder` calculations.
(if (= 4 0)
    (remainder 206 40)
    (gcd (remainder 40 (remainder 206 40))
         (remainder (remainder 206 40) (remainder 40 (remainder 206 40)))))

;; a = (remainder 40 (remainder 206 40))
;; b = (remainder (remainder 206 40) (remainder 40 (remainder 206 40)))
(gcd (remainder 40 (remainder 206 40))
     (remainder (remainder 206 40) (remainder 40 (remainder 206 40)))
(if (= (remainder (remainder 206 40) (remainder 40 (remainder 206 40))) 0)
    (remainder 40 (remainder 206 40))
    (gcd (remainder (remainder 206 40) (remainder 40 (remainder 206 40)))
         (remainder (remainder 40 (remainder 206 40))
                    (remainder (remainder 206 40)
                               (remainder 40 (remainder 206 40)))
;; have to evaluate the conditional so we need to calculate the values of the
;; operands.  Performs four `remainder` calculations.
(if (= 2 0)
    (remainder 40 (remainder 206 40))
    (gcd (remainder (remainder 206 40) (remainder 40 (remainder 206 40)))
         (remainder (remainder 40 (remainder 206 40))
                    (remainder (remainder 206 40)
                               (remainder 40 (remainder 206 40))))))

;; a = (remainder (remainder 206 40) (remainder 40 (remainder 206 40)))
;; b = (remainder (remainder 40 (remainder 206 40))
;;                (remainder (remainder 206 40)
;;                           (remainder 40 (remainder 206 40))))
(gcd (remainder (remainder 206 40) (remainder 40 (remainder 206 40)))
     (remainder (remainder 40 (remainder 206 40))
                (remainder (remainder 206 40)
                           (remainder 40 (remainder 206 40)))))
(if (= (remainder (remainder 40 (remainder 206 40))
                  (remainder (remainder 206 40)
                             (remainder 40 (remainder 206 40))))
       0)
    (remainder (remainder 206 40) (remainder 40 (remainder 206 40)))
    (gcd (remainder (remainder 40 (remainder 206 40))
                    (remainder (remainder 206 40)
                               (remainder 40 (remainder 206 40))))
         (remainder (remainder (remainder 206 40)
                               (remainder 40 (remainder 206 40)))
                    (remainder (remainder 40 (remainder 206 40))
                               (remainder (remainder 206 40)
                                          (remainder 40 (remainder 206 40)))))))
;; have to evaluate the conditional so we need to calculate the values of the
;; operands.  Performs seven `remainder` calculations.
(if (= 0 0)
    (remainder (remainder 206 40) (remainder 40 (remainder 206 40)))
    (gcd (remainder (remainder 40 (remainder 206 40))
                    (remainder (remainder 206 40)
                               (remainder 40 (remainder 206 40))))
         (remainder (remainder (remainder 206 40)
                               (remainder 40 (remainder 206 40)))
                    (remainder (remainder 40 (remainder 206 40))
                               (remainder (remainder 206 40)
                                          (remainder 40 (remainder 206 40)))))))

;; now calculate the result.  Performs four `remainder` calculations.
(remainder (remainder 206 40) (remainder 40 (remainder 206 40)))
2
\end{lstlisting}

\end{document}




%%% Local Variables:
%%% mode: latex
%%% TeX-master: t
%%% End:
