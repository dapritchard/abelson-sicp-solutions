\documentclass{article}

\usepackage[margin=2.75cm]{geometry}
\usepackage[fleqn]{amsmath}
\usepackage{amsthm}
\usepackage{amsfonts}
\usepackage{mathtools}
\usepackage{listings}
\usepackage[x11names]{xcolor}
\usepackage{titlesec}
\usepackage[hidelinks]{hyperref}

% define a global font for listings environments
\lstset{basicstyle=\fontsize{9.5}{10.5}\fontfamily{pcr}\selectfont}

% define a listings environment for Scheme
\lstdefinestyle{scheme}{%
  frame=lines,
  language=lisp,
  basicstyle=\fontsize{9.5}{10.5}\fontfamily{pcr}\selectfont,
  keywordstyle=\color{Blue1},
  commentstyle=\color{DarkOrchid4},
  morekeywords={define, if},
  deletekeywords={count, gcd}
}

% redefine section headers size
\titleformat{\section}[hang]
{\bfseries}
{}
{0em}
{}
% section and subsection spacing
\titlespacing*{\section}{0pt}{10mm plus 1ex minus .2ex}{4mm plus .2ex}




\begin{document}

% question ---------------------------------------------------------------------

\noindent \textbf{Exercise 1.35:} Show that the golden ratio $\phi$ (see Section
1.2.2) is a fixed point of the transformation $x \mapsto 1 + 1/x$, and use this
fact to compute $\phi$ by means of the \lstinline{fixed-point} procedure.

\noindent \hrulefill



\section{Fixed points of functions}

A number $x$ is called a \emph{fixed point} of a function $f$ if $x$ satisfies
the equation $f(x) = x$.  Graphically, these are exactly those points where the
graph of $f$, whose equation is $y = f(x)$, crosses the diagonal, whose equation
is $y = x$.  Clearly, a function might not have any fixed points, or it may have
one or more fixed points.

It turns out that for some choices of $f$ and a starting point $x_0$, that
repeated applications of $f$ to $x_0$ will converge to a fixed point.  Formally,
if we define $x_k = f(x_{k - 1})$ for such a choice of $f$ and
$k = 1, 2, \dots$, then $\lim_{k \rightarrow \infty} x_k = x^{*}$ for some fixed
point $x^{*}$.  Thus we can think of fixed points as ``points of stability'' for
a given model.  For example if the amount of bacteria in a given organism grows
and shrinks over time according to such a function, then this quantity will
eventually converge to a stable value.

It can be shown (c.f.\
\url{http://math.mit.edu/classes/18.01/F2011/lecture3.pdf} or
\url{https://en.wikipedia.org/wiki/Fixed_point_(mathematics)}) that if $f$ is
continuously differentiable in an open neighbourhood of a fixed point $x^{*}$,
and $|f^{\prime}(x^{*})| < 1$, then for a starting point $x_0$ close enough to
$x^{*}$ then repeated application of $f$ to $x_0$ will converge to $x^{*}$.
Conversely, if $|f^{\prime}(x^{*})| > 1$, then repeated application of $f$ to
$x_0$ will \emph{not} converge to $x^{*}$.  Functions with
$|f^{\prime}(x^{*})| = 1$ need to be evaluated on a case-by-case basis.

At the current time I'm not aware of any results for non-continuously
differentiable functions.




\section{Showing that $\phi$ is a fixed point}

\noindent Recall from Section 1.2.2 that $\phi = (1 + \sqrt{5}) / 2$.  Let us
define a function $f: \mathbb{R} \mapsto \mathbb{R}$ with $f(x) = 1 + 1 / x$.
Then we have
\begin{align*} \MoveEqLeft
  f(\phi) = 1 + \left( \frac{1 + \sqrt{5}}{2} \right)^{-1} \\
  &= 1 + \frac{2}{1 + \sqrt{5}} \\[1ex]
  &= \frac{3 + \sqrt{5}}{1 + \sqrt{5}} \\[1ex]
  &= \frac{3 + \sqrt{5}}{1 + \sqrt{5}} \,\cdot \frac{1 - \sqrt{5}}{1 - \sqrt{5}} \\[1ex]
  &= \frac{1 + \sqrt{5}}{2}
\end{align*}




\section{Numerically calculating the value of $\phi$}

Let us define $f$ to be the function given in the prompt (i.e.
$f(x) = (1 + 1/x)$), then we note that $f^{\prime}(x) = -1 / x^2$.  Since the
absolute value of $f^{\prime}(x)$ is less than 1 for values of $x$ in an open
interval around $\phi$, it follows that repeated applications of this $f$ will
converge to $\phi$ for starting values close enough to $\phi$.

The following code sets up the fixed point procedure and function to be
evaluated, and approximates $\phi$ using several different starting points.

\vspace{2mm}
\begin{lstlisting}[style=scheme]
;; a measure of closeness to be used as a parameter for defining when
;; convergence has occurred
(define tolerance 0.00001)


;; search for a fixed point for the function `f`, starting with the value of
;; `first-guess`.  Note that this function should probably have a stopping
;; criterion for function / value pairs for which onvergence does not occur.
(define (fixed-point f first-guess)
  (define (close-enough? v1 v2)
    (< (abs (- v1 v2)) tolerance))
  (define (try guess)
    (let ((next (f guess)))
      (if (close-enough? guess next)
          next
          (try next))))
  (try first-guess))


;; define the function given in the prompt
(define (f x)
  (+ 1 (/ 1 x)))


;; each of these starting points converges to phi, so for this problem the
;; starting value need not need to be close at all to the fixed point to achieve
;; convergence
(fixed-point f     2.0)
(fixed-point f  1000.0)
(fixed-point f -1000.0)
\end{lstlisting}



\end{document}




%%% Local Variables:
%%% mode: latex
%%% TeX-master: t
%%% End:
