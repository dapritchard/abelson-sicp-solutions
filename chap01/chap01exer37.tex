\documentclass{article}

\usepackage[margin=3cm]{geometry}
\usepackage[fleqn]{amsmath}
\usepackage{amsthm}
\usepackage{amsfonts}
\usepackage{mathtools}
\usepackage{listings}
\usepackage[x11names]{xcolor}
\usepackage{titlesec}
\usepackage[hidelinks]{hyperref}
\usepackage{enumitem}


% define a global font for listings environments
\lstset{basicstyle=\fontsize{9.5}{10.5}\fontfamily{pcr}\selectfont}

% define a listings environment for Scheme
\lstdefinestyle{scheme}{%
  frame=lines,
  language=lisp,
  basicstyle=\fontsize{9.5}{10.5}\fontfamily{pcr}\selectfont,
  keywordstyle=\color{Blue1},
  commentstyle=\color{DarkOrchid4},
  morekeywords={define, if},
  deletekeywords={count, gcd}
}

% redefine section headers size
\titleformat{\section}[hang]
{\bfseries}
{}
{0em}
{}
% section and subsection spacing
\titlespacing*{\section}{0pt}{10mm plus 1ex minus .2ex}{4mm plus .2ex}




\begin{document}

% question ---------------------------------------------------------------------

\noindent \textbf{Exercise 1.37:}

\begin{enumerate}[label=\alph*.]
\item An infinite ``continued fraction'' is an expression of the form
  \begin{equation*}
    f = \frac{N_1}{
      D_1 + \frac{N_2}{
        D_2 + \frac{N_3}{D_3 + \dots}}}
  \end{equation*}
  As an example, one can show that the infinite continued fraction expansion
  with the $N_i$ and the $D_i$ all equal to 1 produces $1/\phi$, where $\phi$ is
  the golden ratio (described in Section 1.2.2).  One way to approximate an infinite
  continued fraction is to truncate the expansion after a given number of terms.
  Such a truncation -- a so-called finite continued fraction ``$k$-term finite
  continued fraction'' -- has the form
  \begin{equation*}
    f_k = \frac{N_1}{
      D_1 + \frac{N_2}{
        \dots + \frac{N_k}{D_k}}}
  \end{equation*}
  Suppose that \lstinline{n} and \lstinline{d} are procedures of one argument
  (the term index $i$) that return the $N_i$ and $D_i$ of the terms of
  the continued fraction.  Define a procedure \lstinline{cont-frac} such that evaluating
  \lstinline{(cont-frac n d k)} computes the value of the $k$-term finite continued
  fraction.  Check your procedure by approximating $1/\phi$ using
\begin{lstlisting}[style=scheme, frame=none]
   (cont-frac (lambda (i) 1.0)
              (lambda (i) 1.0)
              k)
\end{lstlisting}
  for successive values of $k$.  How large must you make $k$ in order to
  get an approximation that is accurate to 4 decimal places?

\item If your `cont-frac` procedure generates a recursive process, write one
  that generates an iterative process.  If it generates an iterative
  process, write one that generates a recursive process.

\end{enumerate}

\noindent \hrule




% deriving the form for phi ----------------------------------------------------

\section{Continued fraction expression for $\boldsymbol{\phi}$}

Recall from Section 1.2.2 that $\phi^2 = \phi + 1$.  If we divide both sides by
$\phi$, then we obtain $\phi = 1 + \frac{1}{\phi}$.  Recursively replacing
$\phi$ in the denominator of the right-hand side yields
\begin{equation*}
  1 + \frac{1}{\phi} = 1 + \frac{1}{
    1 + \frac{1}{\phi}},
\end{equation*}
which of course simplifies to
\begin{equation*}
  \frac{1}{\phi} = \frac{1}{
    1 + \frac{1}{\phi}}.
\end{equation*}
Again recursively replacing $\phi$ on the right-hand side yields
\begin{equation*}
  \frac{1}{\phi} = \frac{1}{
    1 + \frac{1}{
      1 + \frac{1}{\phi}}},
\end{equation*}
and so on.  We can use proof-by-induction to establish that $N_k = D_k = 1$ for
arbitrary $k$ in the continued fraction expression, as claimed in the prompt.




% \section{Showing that $\phi$ is a fixed point}

% \noindent Recall from Section 1.2.2 that $\phi = (1 + \sqrt{5}) / 2$.  Let us
% define a function $f: \mathbb{R} \mapsto \mathbb{R}$ with $f(x) = 1 + 1 / x$.
% Then we have
% \begin{align*} \MoveEqLeft
%   f(\phi) = 1 + \left( \frac{1 + \sqrt{5}}{2} \right)^{-1} \\
%   &= 1 + \frac{2}{1 + \sqrt{5}} \\[1ex]
%   &= \frac{3 + \sqrt{5}}{1 + \sqrt{5}} \\[1ex]
%   &= \frac{3 + \sqrt{5}}{1 + \sqrt{5}} \,\cdot \frac{1 - \sqrt{5}}{1 - \sqrt{5}} \\[1ex]
%   &= \frac{1 + \sqrt{5}}{2}
% \end{align*}




\end{document}




%%% Local Variables:
%%% mode: latex
%%% TeX-master: t
%%% End:
