\documentclass{article}

\usepackage[margin=3cm]{geometry}
\usepackage[fleqn]{amsmath}
\usepackage{amsthm}
\usepackage{amsfonts}
\usepackage{mathtools}
\usepackage{listings}
\usepackage[x11names]{xcolor}
\usepackage{titlesec}
\usepackage[hidelinks]{hyperref}
\usepackage{enumitem}


% define a global font for listings environments
\lstset{basicstyle=\fontsize{9.5}{10.5}\fontfamily{pcr}\selectfont}

% define a listings environment for Scheme
\lstdefinestyle{scheme}{%
  frame=lines,
  language=lisp,
  basicstyle=\fontsize{9.5}{10.5}\fontfamily{pcr}\selectfont,
  keywordstyle=\color{Blue1},
  commentstyle=\color{DarkOrchid4},
  morekeywords={define, if},
  deletekeywords={count, gcd}
}

% redefine section headers size
\titleformat{\section}[hang]
{\bfseries}
{}
{0em}
{}
% section and subsection spacing
\titlespacing*{\section}{0pt}{10mm plus 1ex minus .2ex}{4mm plus .2ex}




\begin{document}

% question ---------------------------------------------------------------------

\noindent \textbf{Exercise 1.37:}

\begin{enumerate}[label=\alph*.]
\item An infinite ``continued fraction'' is an expression of the form
  \begin{equation*}
    f = \frac{N_1}{
      D_1 + \frac{N_2}{
        D_2 + \frac{N_3}{D_3 + \dots}}}
  \end{equation*}
  As an example, one can show that the infinite continued fraction expansion
  with the $N_i$ and the $D_i$ all equal to 1 produces $1/\phi$, where $\phi$ is
  the golden ratio (described in Section 1.2.2).  One way to approximate an infinite
  continued fraction is to truncate the expansion after a given number of terms.
  Such a truncation -- a so-called finite continued fraction ``$k$-term finite
  continued fraction'' -- has the form
  \begin{equation*}
    f_k = \frac{N_1}{
      D_1 + \frac{N_2}{
        \dots + \frac{N_k}{D_k}}}
  \end{equation*}
  Suppose that \lstinline{n} and \lstinline{d} are procedures of one argument
  (the term index $i$) that return the $N_i$ and $D_i$ of the terms of
  the continued fraction.  Define a procedure \lstinline{cont-frac} such that evaluating
  \lstinline{(cont-frac n d k)} computes the value of the $k$-term finite continued
  fraction.  Check your procedure by approximating $1/\phi$ using
\begin{lstlisting}[style=scheme, frame=none]
   (cont-frac (lambda (i) 1.0)
              (lambda (i) 1.0)
              k)
\end{lstlisting}
  for successive values of $k$.  How large must you make $k$ in order to
  get an approximation that is accurate to 4 decimal places?

\item If your `cont-frac` procedure generates a recursive process, write one
  that generates an iterative process.  If it generates an iterative
  process, write one that generates a recursive process.

\end{enumerate}

\noindent \hrule




% deriving the form for phi ----------------------------------------------------

\section{Continued fraction expression for $1 / \phi$}

Recall from Section 1.2.2 that $\phi^2 = \phi + 1$.  If we divide both sides by
$\phi$, then we obtain $\phi = 1 + \frac{1}{\phi}$.  Recursively replacing
$\phi$ in the denominator of the right-hand side yields
\begin{equation*}
  1 + \frac{1}{\phi} = 1 + \frac{1}{
    1 + \frac{1}{\phi}},
\end{equation*}
which then simplifies to
\begin{equation*}
  \frac{1}{\phi} = \frac{1}{
    1 + \frac{1}{\phi}}.
\end{equation*}
Once again replacing $\phi$ on the right-hand side yields
\begin{equation*}
  \frac{1}{\phi} = \frac{1}{
    1 + \frac{1}{
      1 + \frac{1}{\phi}}},
\end{equation*}
and we can continue to do this indefinitely.  We can use proof-by-induction to
establish that $N_k = D_k = 1$ for arbitrary $k$ in the continued fraction
expression, as claimed in the prompt.




% part a (recursive version) ---------------------------------------------------

\section{Creating a recursive version of the $k$-term finite continued fraction}

We can either count up starting from 1 towards $k$ or count down starting from
$k$ towards 1.  It turns out that starting from 1 and counting up naturally
leads to a recursive version.

\begin{lstlisting}[style=scheme]
;; linear version.  Calculates a k-term finite continued fraction where the
;; values of `N_i` and `D_i` are given by evaluating `n-fcn` and `d-fcn` at i.
;; Note that we add by 0.0 during one of the calculations to ensure the use of
;; floating point operations (instead of fractional).
(define (cont-frac n-fcn d-fcn k)
  (define (i-th-level i)
    (if (> i k)
	0
	(/ (n-fcn i)
	   (+ 0.0
	      (d-fcn i)
	      (i-th-level (1+ i))))))
  (i-th-level 1))
\end{lstlisting}




% part b (iterative version) ---------------------------------------------------

\section{Creating an iterative version of the $k$-term finite continued fraction}

Counting down from $k$ naturally leads to an iterative version.

\begin{lstlisting}[style=scheme]
;; iterative version.  Calculates a k-term finite continued fraction where the
;; values of `N_i` and `D_i` are given by evaluating `n-fcn` and `d-fcn` at i.
;; Note that we add by 0.0 during one of the calculations to ensure the use of
;; floating point operations (instead of fractional).
(define (cont-frac-iter n-fcn d-fcn k)
  (define (calc-inner-term n-k d-k inner-result)
    (/ n-k (+ 0.0 d-k inner-result)))
  (define (i-th-level i inner-result)
    (if (= 0 i)
	inner-result
	(i-th-level (1- i)
		    (calc-inner-term (n-fcn i)
				     (d-fcn i)
				     inner-result))))
  (i-th-level (1- k)
	      (/ (n-fcn k)
		 (d-fcn k))))
\end{lstlisting}




% helper functions -------------------------------------------------------------

\section{Helper functions}

Here we define a function \lstinline{find-first-close} for later use to measure
how many terms are necessary to closely approximate the desired values.

\begin{lstlisting}[style=scheme]
(define MAX-ITER 1000000)
(define (1- x) (- x 1))

;; test whether the difference between `actual` and `target` is less than `eps`
(define (close-enough? actual target eps)
  (< (abs (- actual target)) eps))


;; repeatedly evaluate `f`, a function that takes a single integer input, until
;; the difference beteen the result of `f` and `target` is less than `eps`.  The
;; values passed to `f` start at 1 and increase by 1 each time, and the return
;; value is the number of attempts that are necessary for an evaluation clsoe to
;; `target` is found.
(define (find-first-close f target eps)
  (define (try i)
    (cond ((> i MAX-ITER) (error "sequence did not converge"))
          ((close-enough? (f i) target eps) i)
          (#t (try (1+ i)))))
  (try 1))
\end{lstlisting}




% test cases -------------------------------------------------------------------

\section{Test cases}

Check that our function approximating $1 / \phi$ yields a reasonable result, and
measure how many terms are needed to get a good approximation.

\begin{lstlisting}[style=scheme]
;; the continued fraction expression for 1 / phi
(define (n-1-fcn i) 1)
(define (d-1-fcn i) 1)
(define (frac-phi-inv k)
  (cont-frac-iter n-1-fcn d-1-fcn k))


;; 1 / phi:  should be approximately 0.6180339887498949
(frac-phi-inv 100)

;; the size of `k` needed to approximate the value of 1 / phi to 4 decimal places
(find-first-close frac-phi-inv 0.6180339887498949 0.00001)
\end{lstlisting}





\end{document}




%%% Local Variables:
%%% mode: latex
%%% TeX-master: t
%%% End:
