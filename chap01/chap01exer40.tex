\documentclass{article}

\usepackage[margin=2.7cm]{geometry}
\usepackage[fleqn]{amsmath}
\usepackage{amsthm}
\usepackage{amsfonts}
\usepackage{mathtools}
\usepackage{listings}
\usepackage[x11names]{xcolor}
\usepackage{titlesec}
\usepackage[hidelinks]{hyperref}

% define a little O symbol
\newcommand{\littleO}{\bgroup \scriptstyle \boldsymbol{\mathcal{O}} \egroup}

% define a global font for listings environments
\lstset{basicstyle=\fontsize{9.5}{10.5}\fontfamily{pcr}\selectfont}

% define a listings environment for Scheme
\lstdefinestyle{scheme}{%
  frame=lines,
  language=lisp,
  basicstyle=\fontsize{9.5}{10.5}\fontfamily{pcr}\selectfont,
  keywordstyle=\color{Blue1},
  commentstyle=\color{DarkOrchid4},
  morekeywords={define, if},
  deletekeywords={count, gcd}
}

% redefine section headers size
\titleformat{\section}[hang]
{\bfseries}
{}
{0em}
{}
% section and subsection spacing
\titlespacing*{\section}{0pt}{10mm plus 1ex minus .2ex}{4mm plus .2ex}




\begin{document}

% question ---------------------------------------------------------------------

\noindent \textbf{Exercise 1.40:} Define a procedure \lstinline{cubic} that can
be used together with the \lstinline{newtons-method} procedure in expressions of
the form \lstinline{(newtons-method (cubic a b c) 1)} to approximate zeros of
the cubic $x^3 + ax^2 + bx + c$.

\noindent \hrulefill




\section{Relationship between Newton's method and fixed points}

Recall that the first order Taylor series approximation of a function $f$ about
a point $a$ is given by
\begin{equation}
  f(x) = f(a) + f^{\prime}(a)\, (x -a) + \littleO( |x - a| ).
\end{equation}
If $x$ is a root for $f$ then the left-hand side becomes 0, so that by
rearranging the terms and when $a$ is sufficiently close to $x$ we obtain
\begin{equation}
  \label{eq:taylor-for-root}
  x \approx a - \frac{f(a)}{f^{\prime}(a)}.
\end{equation}
This gives us one way to derive Newton's method.  If we consider the sequence
$x_1, x_2, \dots$ given by
\begin{equation}
  \label{eq:newton}
  x_n = x_{n - 1} - \frac{f(x_{n - 1})}{f^{\prime}(x_{n - 1})},
\end{equation}
for some choice of $x_0$, then the idea is that the approximation given in
(\ref{eq:taylor-for-root}) will become arbitrarily accurate due to the vanishing
nature of the $\littleO( |x_n - x_{n - 1}| )$ term, under certain regularity
conditions.

Next, we show the relationship of Newton's method to the fixed-point method.
Consider some arbitrary function $f^{*}$, and let us define
\begin{equation}
  \label{eq:fixed-point}
  g(x) = x - \frac{f(x)}{f^{*}(x)}.
\end{equation}
Note for $x$ to be a fixed point of $g$, it is necessary and sufficient that
$f(x) = 0$ and $f^{*}(x) \ne 0$.  In other words, for a function of the form
given by $g$, finding fixed points of $g$ is equivalent to finding roots of $f$.
Furthermore, we note that the right-hand side of (\ref{eq:newton}) is just a
special case of (\ref{eq:fixed-point}).  Thus we can view the Taylor series
approximation as providing us a good choice of $f^{*}$, although this
demonstrates it is not the only choice.




\section{Code for Newton's method}

In the following code block we supply the functions from Sections 1.3.3 and
1.3.4 needed for \lstinline{newtons-}\allowbreak{}\lstinline{method}.

\begin{lstlisting}[style=scheme]
;; a measure of closeness to be used as a parameter for defining when
;; convergence has occurred
(define tolerance 0.00001)
;; the size of the change in used to approximate the derivative of a function
(define dx 0.00001)


;; search for a fixed point for the function `f`, starting with the value of
;; `first-guess`.  Note that this function should probably have a stopping
;; criterion for function / value pairs for which onvergence does not occur.
(define (fixed-point f first-guess)
  (define (close-enough? v1 v2)
    (< (abs (- v1 v2)) tolerance))
  (define (try guess)
    (let ((next (f guess)))
      (if (close-enough? guess next)
	  next
	  (try next))))
  (try first-guess))


;; uses Newton's method to find a fixed point for function `g`, using an initial
;; value given by `guess`
(define (newtons-method g guess)
  (fixed-point (newton-transform g) guess))


;; takes an input function `g` and returns a function with input `x` that
;; calculates `x - (f(x) / (d/dx f(x)))`
(define (newton-transform g)
  (lambda (x)
    (- x (/ (g x) ((deriv g) x)))))


;; takes an input function `g` and returns a function with input `x` that
;; calculates an approximation of the derivative of `g` evaluated at `x`
(define (deriv g)
  (lambda (x)
    (/ (- (g (+ x dx)) (g x))
       dx)))
\end{lstlisting}




\section{Creating the cubic polynomial function}

Next, we define the \lstinline{cubic} function specified in the prompt.  We note
that since the first coefficient always has a value of 1, this function goes to
$-\infty$ and $\infty$ in the limit as the input goes to $-\infty$ and $\infty$,
respectively.  This means that there is guaranteed to be at least one root for
any choices of \lstinline{a}, \lstinline{b}, and \lstinline{c}.
\begin{lstlisting}[style=scheme]
;; takes values for `a`, `b`, and `c`, and returns a function that takes an
;; input `x` and returns the value given by `x^3 + ax^2 + bx + c`.
(define (cubic a b c)
  (lambda (x)
    (+ (* x x x)
       (* a x x)
       (* b x )
       c)))
\end{lstlisting}




\section{Test cases}

Now consider a few test cases to ensure that the Newton's method is able to find
a roots for several combinations of inputs.

\begin{lstlisting}[style=scheme]
;; should have a value of 1
(newtons-method (cubic 1 -2 0) 1.0)

;; find the value and verify that it is a root
(newtons-method (cubic 1 -3 0) 1.0)
((cubic 1 -3 0) 1.302775637742353)

;; find the value and verify that it is a root
(newtons-method (cubic 1 -3 2) 1.0)
((cubic 1.0 -3 2) -2.5115471416945176)
\end{lstlisting}

\end{document}


%%% Local Variables:
%%% mode: latex
%%% TeX-master: t
%%% End:
